%----------------------------------------------------------------------------------------
% Einführung in das Wissenschaftliche Arbeiten 
% Report LaTeX template (English)
% Interactive Graphics and Simulation Group
% University of Innsbruck
%----------------------------------------------------------------------------------------

\documentclass[11pt,a4paper]{article}

%----------------------------------------------------------------------------------------
% Include required packages
\usepackage{graphicx}
\usepackage{amsmath}

\usepackage[english]{babel}
%\usepackage[ngerman]{babel}

\usepackage[utf8x]{inputenc}
\usepackage[T1]{fontenc}

\usepackage[left=2.7cm, right=2.7cm, top=3cm]{geometry}

%----------------------------------------------------------------------------------------
% Start document

\begin{document}


%----------------------------------------------------------------------------------------
% Title page
%----------------------------------------------------------------------------------------

\begin{titlepage} % User-defined title page

\begin{center}
\includegraphics[width=1.2cm]{images/uibk}

\begin{large}
Leopold-Franzens-Universität Innsbruck\\[5mm]
Institute of Computer Science\\
Interactive Graphics and Simulation Group\\[25mm]
\end{large}

{\LARGE \bf Raytracing Project}

Advanced Computer Graphics\\ 
Documentation\\[15mm]

Phillip Mildenberger\\
Stefan Spiss\\
Cem Okulmus\\[35mm]

advised by\\
Savoye Yann Pierre, PhD\\[10mm]

\vfill

Innsbruck, \today
\end{center}

\end{titlepage}


%----------------------------------------------------------------------------------------
% Main body
%----------------------------------------------------------------------------------------

\section{Introduction}
\label{sec:intro}

For this project we had to implement certain advanced features using the smallPT code as base. Every student in one group had to implement at least one main feature. In our group we chose Motion Blur (by Cem Okulmus), Depth of Field (by Stefan Spiss) and Specular/Texture/Normal Mapping (by Phillip Mildenberger). Additionally in our project certain features were implemented, such as: Acceleration Structures (by Phillip Mildenberger), simple Camera Control (by Stefan Spiss)  

\section{Implementation Detail}
In this section we will give a short explanation on every implemented feature. All of these are extensions of the smallPT code. 

\subsection{Motion Blur}
As was explained during the lecture, the basic idea behind this effect is that we want to distribute the rays not in space but in time. For this we usually also want to have some sort of animation, otherwise the effect wouldn't show. This method was first detailed by Cook et ali. [] 
Since smallPT also uses a distribution over space, only an extension of this was needed.
The rays are uniformally distributed over time, with equal weight for all time steps. 

\subsection{Depth of Field}
In standard smallPT the pinhole camera model is used. So everything in the scene is in focus. Normal cameras always have a Depth of Field, which means that only objects in a specific distance are in focus. Everything else is blurred depending on the distance to the focal point. In the project implementation the Thin-Lens model is used. For this the ray direction is changed. We only not from one camera position, but distributed over the aperture circle. As target for the rays, the focal plane is used instead of the image plane. 


\subsection{Specular/Texture/Normal Mapping}

UV mapping from obj-files is supported, including import of color, specular and normal data.
For specular mapping a new material had to be created: Glossy surface. It's used to supply specular data with a texture. The normal mapping is changing the intersection normal for depending on the texture for the surface. 
For every intersection we return a special intersection structure, additionally containing the needed data for specular/normal/texture mapping. 

\subsection{Acceleration Structures}

To accelerate the intersection of complex triangle-meshes in smallPT, we implemented different acceleration structures. 
The first is just simple approach: A bounding box is put around every object, and the intersection function first checks every bounding box and only continues if one was hit.
The second is a  Bounding Volume Hierarchy (BVH). An octree is used to build up the hierarchy recursively: Every object is inserted using its centroid. From these objects, using bottom-up traversal, the BVH is constructed. 

\section{Discussion}
\label{sec:discn}

An example image is provided in Fig.~\ref{fig:figure1}.
Equation~\eqref{eq:exp1} and equation~\eqref{eq:exp2} are further examples.
Finally, on page~\pageref{tab:table1} is an example of a table.

Using the definition of
\begin{equation}
  \label{eq:exp1}
  \sum\limits_{i} x_{i} = 0,
\end{equation}

we can define 

\begin{equation}
  \label{eq:exp2}
  f_{int} + f_{ext} = 0.
\end{equation}

A figure follows.

\begin{figure}[h!]
  \centering
  %\includegraphics[width=5cm]{image}
  \LaTeX
  \caption{Text describing the figure.}
  \label{fig:figure1}
\end{figure}

\begin{table}[b!]
  \centering
  \begin{tabular}{|c|c|c|c|}
\hline
number & $axis$ & value  \\   \hline
1 &   $x$     & 40    \\   \hline
2 &   $y$     & 80    \\   \hline
3 &   $z$     & 50    \\   \hline
  \end{tabular}
  \caption{Example table.}
  \label{tab:table1}
\end{table}

Random text: Lorem ipsum dolor sit amet, consetetur sadipscing elitr, sed diam nonumy 
eirmod tempor invidunt ut labore et dolore magna aliquyam erat, sed diam voluptua. At 
vero eos et accusam et justo duo dolores et ea rebum. Stet clita kasd gubergren, no sea 
takimata sanctus est Lorem ipsum dolor sit amet. Lorem ipsum dolor sit amet, consetetur 
sadipscing elitr, sed diam nonumy eirmod tempor invidunt ut labore et dolore magna 
aliquyam erat, sed diam voluptua. At vero eos et accusam et justo duo dolores et ea 
rebum. Stet clita kasd gubergren, no sea takimata sanctus est Lorem ipsum dolor sit amet.

\section{Conclusion}
\label{sec:concl}

Example literatur is available in \cite{shirley:2009} and \cite{seiler:2013}.

Random text: Lorem ipsum dolor sit amet, consetetur sadipscing elitr, sed diam nonumy 
eirmod tempor invidunt ut labore et dolore magna aliquyam erat, sed diam voluptua. At 
vero eos et accusam et justo duo dolores et ea rebum. Stet clita kasd gubergren, no sea 
takimata sanctus est Lorem ipsum dolor sit amet. Lorem ipsum dolor sit amet, consetetur 
sadipscing elitr, sed diam nonumy eirmod tempor invidunt ut labore et dolore magna 
aliquyam erat, sed diam voluptua. At vero eos et accusam et justo duo dolores et ea 
rebum. Stet clita kasd gubergren, no sea takimata sanctus est Lorem ipsum dolor sit amet.

    

%----------------------------------------------------------------------------------------
% Bibliography
%----------------------------------------------------------------------------------------

\bibliographystyle{plain}
\bibliography{EWA_literature}

\end{document}
